\newcommand{\institut}{Institut f\"ur Energie und Automatisiertungstechnik}
\newcommand{\fachgebiet}{Elektronische Mess- und Diagnosetechnik}
\newcommand{\veranstaltung}{Praktikum Messdatenverarbeitung}
\newcommand{\pdfautor}{Dirk Babendererde (321 836), Thomas Kapa (325 219), Magdalene Busuru (319 433)}
\newcommand{\autor}{Dirk Babendererde (321 836)\\ Thomas Kapa (325 219)\\ Magdalene Busuru (319 433)}
\newcommand{\pdftitle}{Praktikum Messdatenverarbeitung Termin 2}
\newcommand{\prototitle}{Praktikum Messdatenverarbeitung \\ Termin 2}
\newcommand{\aufgabe}{}



\newcommand{\gruppe}{Gruppe: G2 Fr 08-10}
\newcommand{\betreuer}{Betreuer: J\"urgen Funk}


\input{../../packages/tu_header_8}


% \lstlistoflistings
\definecolor{darkgray}{rgb}{0.95,0.95,0.95}
\lstset{language=c}
\lstset{inputencoding=utf8}
%\lstset{extendedchars=true} % Umlaute an der richtigen stelle und nicht am Anfang ausgeben
\lstset{backgroundcolor=\color{darkgray}}
\lstset{numbers=left, numberstyle=\tiny, stepnumber=1, numbersep=7pt, breaklines=true}
\lstset{keywordstyle=\color{red}\bfseries\emph}
\lstset{
breaklines,
numbers=left,
frame=single,
xleftmargin=-2cm,
xrightmargin=-1.5cm
}
% enables UTF-8 in source code: (dirty, dirty hack)
\lstset{literate=
    %Deutsch
    {ä}{{\"a}}1 {ö}{{\"o}}1 {ü}{{\"u}}1 {Ä}{{\"A}}1 {Ö}
    {{\"O}}1 {Ü}{{\"U}}1 {ß}{\ss}1
    %Türkisch
    {â}{{\^{a}}}1 {Â}{{\^{A}}}1 {ç}{{\c{c}}}1 {Ç}{{\c{C}}}1 {ğ}{{\u{g}}}1 {Ğ}{{\u{G}}}1 {ı}{{\i}}1 {İ}{{\.{I}}}1 {ö}{{\"o}}1 {Ö}{{\"O}}1 {ş}{{\c{s}}}1
    {Ş}{{\c{S}}}1 {ü}{{\"u}}1 {Ü}{{\"U}}1
    %Polish
    {ą}{{\k{a}}}1 {ć}{{\'c}}1 {ę}{{\k{e}}}1 {ł}{{\l{}}}1 {ń}{{\'n}}1 {ó}{{\'o}}1 {ś}{{\'s}}1 {ż}{{\.z}}1 {ź}{{\'z}}1 {Ą}{{\k{A}}}1 {Ć}{{\'C}}1
    {Ę}{{\k{E}}}1 {�}{{\L{}}}1 {Ń}{{\'N}}1 {Ó}{{\'O}}1 {Ś}{{\'S}}1 {Ż}{{\.Z}}1 {Ź}{{\'Z}}1
    %Spanish
    {á}{{\'a}}1 {é}{{\'e}}1 {í}{{\'i}}1 {ó}{{\'o}}1 {ú}{{\'u}}1 {ñ}{{\~n}}1
}

%     \lstinputlisting{./praktikum6.sce}



%---------------------------------------------------------------------
%---------------------------------------------------------------------
%---------------------------------------------------------------------

\section{Vorbereitung}
\begin{quote}
    \subsection{Abtastrate}
    \begin{quote}
        
        Die CPU-Taktfreqenz wird (wie eingestellt) durch 32 geteilt um den ADU-Takt zu erhalten. Nun braucht der ADU 13 Takte für eine Umsetzung.
        Daraus folgt:
        
        $\frac{1}{\frac{32 \p 13}{7,3 \p 10^6}} \approx 17,5 kHz$
        
    \end{quote}

    \subsection{Sourcecode}
    \lstinputlisting[
        caption={C-script},
        label=lst:c-vorbereitung]
        {./C/adu.c}
            


    
    
\end{quote}







\bq
\lstinputlisting{./C/adc.c}
\eq

\end{document}

